\section{Ergebnisse und Berechnungen}
\label{sec:ergebnisse}
%Tabelle START

	\vspace*{-2.5mm}
	\renewcommand{\arraystretch}{1.2}
	\begin{table}[h!]
		\centering
		\caption{Messwerte zur Titerbestimmung}
		\label{tab:MesswerteTiterbestimmung}
			%\resizebox{17cm}{!}{
		\begin{tabulary}{\textwidth}{|C|C|C|C|C|C|C|}
		\hline
		\multicolumn{7}{|c|}{Automatisch }\\
			\hline
			\textbf{Probengröße Wasser} [\si{\gram}]& \textbf{Konz.} [\si{\milli\gram\per\milli\liter}]& \textbf{Titer} &\textbf{DRIFT} [\si{\micro\gram\per\minute}]&\textbf{DRIFT} [\si{\micro\liter\per\minute}]&\textbf{Dauer} [min]&\textbf{V$_{EQ}$} [\si{\milli\liter}]\\
			\hline
			0,005& 4,81040 &0,962&4,3&0,7&1,55&1,0405
			\\
			\hline
			0,005& 5,18954 &1,038&1,1&0,2&1,37&0,96375
			\\
			\hline
			0,005& 5,68201&1,136&&&&\\
			\hline
			\hline
			\textbf{Mittelwert}&5,227&1,045 &&&&\\
			\cline{1-3}
			\textbf{Standard\-abweichung}&0,437&0,0874
			&&&&\\
			\cline{1-3}
			\textbf{rel.Strd.\-abweichung}&8,361\%&8,361\% &&&&\\
			\hline
		
		\end{tabulary}
	%	}
	\end{table}
	\FloatBarrier 
	% \vspace*{-2.5mm}
	%Tabelle ENDE
   
%Tabelle START
 
  \vspace*{-2.5mm}
 \renewcommand{\arraystretch}{1.2}
 \begin{table}[h!]
 	\centering
 	\caption{Messwerte zur Isopropanol-Probe}
 	\label{tab:MesswerteIsopropanol}
 %	\resizebox{17cm}{!}{
 	\begin{tabulary}{\textwidth}{|C|C||C|C|C|}
 		\hline
 		\multicolumn{5}{|c|}{Automatisch }\\
 		\hline
 		\textbf{Probengröße Isopropanol} [\si{\gram}]& \textbf{m\% \ce{H2O}}& \textbf{m(\ce{H2O})} [\si{\milli\gram}]&\textbf{DRIFT} [\si{\micro\gram\per\minute}]&\textbf{DRIFT} [\si{\micro\liter\per\minute}]\\
 		\hline
 		0,4037& 0,0036 &0,0145&28,6&5,5\\
 		\cline{1-3}
 		0,3716& 0,3553 &1,2303&&\\
 		\hline
 		\hline
 	\textbf{Mittelwert}&0,1795&0,6674 &&\\
 	\cline{1-3}
 	\textbf{Standard\-abweichung}&0,2487&0,9233&&\\
 	\cline{1-3}
 	\textbf{rel.Strd.\-abweichung}&138,552\%&138,342\% &&\\
 	\hline
 	\hline
 \textbf{Versuchszeit}&\multicolumn{4}{|c|}{1,18\si{\minute}} \\
 \hline
 \end{tabulary}
 	%}
 \end{table}
 \FloatBarrier 
% \vspace*{-2.5mm}
 %Tabelle ENDE


%Tabelle START
\vspace*{-2.5mm}
\renewcommand{\arraystretch}{1.2}
\begin{table}[h!]
	\centering
	\caption{Messwerte zur Isopropanol-Probe/2-Propanol}
	\label{tab:Messwerte2Propanol}
	%	\resizebox{17cm}{!}{
	\begin{tabulary}{\textwidth}{|C|C||C|}
		\hline
		\multicolumn{3}{|c|}{Automatisch }\\
		\hline
		\textbf{Probengröße Isopropanol} [\si{\gram}]& \textbf{m\% \ce{H2O}}& \textbf{m(\ce{H2O})} [\si{\milli\gram}]\\
		\hline
		0,7401& 0,0435&0,3219\\
		\cline{1-3}
		0,08851& 0,0384 &0,3399\\
		
		\cline{1-3}
		0,9157& 0,0431 &0,3947\\
		\cline{1-3}
		0,9442& 0,0379 &0,3579\\
		
		\hline
		\hline
		\textbf{Mittelwert}&0,0407&0,3536 \\
		\cline{1-3}
		\textbf{Standard\-abweichung}&0,0030&0,0311\\
		\cline{1-3}
		\textbf{rel.Strd.\-abweichung}&7,3292\%&8,7855\% \\
		\hline
		
	\end{tabulary}
	%}
\end{table}
\FloatBarrier 
% \vspace*{-2.5mm}
%Tabelle ENDE

%Tabelle START
\vspace*{-2.5mm}
\renewcommand{\arraystretch}{1.2}
\begin{table}[h!]
	\centering
	\caption{Messwerte zur Polyamid-Probe}
	\label{tab:MesswertePoliamid}
	%	\resizebox{17cm}{!}{
	\begin{tabulary}{\textwidth}{|C|C||C|}
		\hline
		\multicolumn{3}{|c|}{Automatisch }\\
		\hline
		\textbf{Probengröße Polyamid} [\si{\gram}]& \textbf{m\% \ce{H2O}}& \textbf{m(\ce{H2O})} [\si{\milli\gram}]\\
		\hline
		0,1996& 2,3217&4,6341\\
		\cline{1-3}
		0,2028& 1,9189 &3,8915\\
		
		\hline
		\hline
		\textbf{Mittelwert}&2,1203&4,2628 \\
		\cline{1-3}
		\textbf{Standard\-abweichung}&0,2848&0,5251\\
		\cline{1-3}
		\textbf{rel.Strd.\-abweichung}&13,4331\%&12,3178\% \\
		\hline
		
	\end{tabulary}
	%}
\end{table}
\FloatBarrier 
% \vspace*{-2.5mm}
%Tabelle ENDE
\subsection{Manuelle Berechnung der relativen Standardabweichung}
Die Berechnung der relativen Standardabweichung von Hand, erfolgt analog der Gleichung (\ref{gl:S_rel}). Die mit dem Taschenrechner erhaltenen Ergebnisse sind in den Tabellen \ref{tab:MesswerteTiterbestimmung} bis \ref{tab:MesswertePoliamid} in der Sektion \glqq Manuell\grqq  \, aufgeführt. 
\begin{flalign}\label{gl:S_rel}
	s_{rel}&=\frac{s}{\bar{x}}\\
	&=\frac{0,2848\%}{2,1203\%}\\
	&=0,134321=\underline{\underline{13,4321\%}}
\end{flalign}

\subsection{Manuelle Titerbestimmung}
Die Berechnung des Titers $t$ der KF-Lösung ist in Gleichung (\ref{gl:titer}) beispielhaft für ein Wertepaar dargestellt. Die verwendeten Werte entstammen der ersten Zeile der Tabelle \ref{tab:MesswerteTiterbestimmung}.
\begin{flalign}\label{gl:titer}
	t&=\frac{c_{ist}}{c_{soll}}\\
	&=\frac{\SI{4,81040}{\milli\gram\per\milli\liter}}{\SI{5,0}{\milli\gram\per\milli\liter}}\\
	&=\underline{\underline{0,96208}}
\end{flalign}
\subsection{Manuelle Bestimmung des Wassergehaltes}