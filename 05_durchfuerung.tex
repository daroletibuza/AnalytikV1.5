\section{Durchführung}
\label{sec:durchfuerung}
Tatsächlich wurde das Praktikum nicht durch die Autoren dieses Protokolls durchgeführt. Aus diesem Grund wird an dieser Stelle die Versuchsdurchführung nur in vereinfachter Form dargestellt.\\
Zunächst würde der Titrator in Betrieb genommen und auf die Titratorverbindung zum Rechner gewartet werden. Parallel dazu würde nun der Spülgasstrom mit Stickstoff eingeschaltet und der Ofen für den Versuchsteil 2 auf \SI{300}{\celsius} vorgeheizt werden. \\
Sind die Vorbereitungen getroffen, würde infolgedessen die Kalibrierung der KF-Lösung mit \SI{5}{\micro \liter} einer Wasserstandardsubstanz erfolgen. Über die ermittelten Messwerte kann somit der Titer der Titrierlösung bestimmt werden.
Als nächstes würden nun zwei Flüssigproben nach ähnlichem Vorgehen auf ihren Wassergehalt untersucht werden.\\
Im zweiten Versuchsteil würde nun für eine Feststoffprobe, über die Ofentechnik, der Wassergehalt bestimmt. Die Ofentemperatur würde hierfür nun auf \SI{200}{\celsius} abgesenkt werden. Danach wird die entsprechende Feststoffprobe in den Ofen gelegt und die Titration gestartet.\\
Im Anschluss daran, würden die entsprechenden Datensätze am PC mit \textsc{LabX} ausgewählt und gedruckt werden.