\section{Diskussion}
\label{sec:diskussion}


\subsection{Bedeutung der Driftkorrektur}
Der Drift wurde als linearer Zusammenhang vom Messgerät durch Kalibrierung erfasst. Die Einbeziehung des Massen- und Volumendrifts erlaubt es die wiederkehrende Abweichung des Messgerätes zu berücksichtigen und damit eine größere Genauigkeit des erhaltenen Ergebnisses. Eine absolute Abweichung wird dabei durch Multiplikation mit der Messdauer erhalten.

\subsection{Bewertung der Feuchtigkeit}

\subsubsection{Isopropanol}
Die Feuchtigkeitsmesswerte des Isopropanols sind in der Tabelle \ref{tab:MesswerteIsopropanol} einzusehen. Der mittlere gemessene Wassergehalt von 0,0407\% ist sehr gering. Isopropanol ist in jedem Verhältnis mit Wasser Mischbar.\cite{isopropanol} Daher ist der Wassergehalt plausibel. 

\subsubsection{Polyamid}

Die Feuchtigkeitsmesswerte des Polyamids sind in der Tabelle \ref{tab:MesswertePoliamid} einzusehen. Der mittlere gemessene Wassergehalt von 2,1203\% ist deutlich höher als beim Isopropanol. 

\subsubsection{Risiken von Wasser bei der Polymerverarbeitung}
Die gebräuchlichsten Alltagspolymere sind Thermoplaste. Sie werden zur Verarbeitung und Umformung meist erwärmt. Enthaltenes Wasser verdampft dabei. Kann es nicht entweichen, so baut sich Druck auf. Dieser Druck könnte Verarbeitungsmaschinen wie etwa Strangpressen mechanisch bis zur Zerstörung belasten. Wasserdampf kann für Korrosion an ungeschützten Stahlteilen sorgen. Austretender Wasserdampf oder herausgeschleuderter heißer Kunststoff kann auch Personen durch Verbrennungen schädigen. Wasser wirkt, als polare Substanz, außerdem als Trennmittel für den Kunststoff. Daher kann enthaltenes Wasser auch für Fehlstellen im Material verantwortlich sein.