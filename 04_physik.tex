\section{Theorie}
\label{sec:theorie}

\subsection{Art der Titration}
Die in diesem Versuch angewendete Art der Titration ist eine Redoxtitration. Dedoxtitration zeichnen sich dadurch aus, dass sie auf einer Redoxreaktion beruhen. Redoxreaktionen sind die chemische Reaktion eines Oxidationsmittels mit einem Reduktionsmittel.\cite{redoxreaktion}
\subsubsection{Konkrete chemische Reaktion}
Die hier angewendete Redoxreaktion beruht auf der Reduktion von Iod. Daher auch der Name Iodometrie. Die Zusammengefasste Reaktionsgelichung ist als Gleichung (\ref{gl:gesamtreaktion}) nachfolgend aufgeführt. 
\begin{equation}\label{gl:gesamtreaktion}
	\ce{I2 + SO2 + 3 RN + CH3OH + H2O -> 2RN*HI + RN*HSO4CH3}
\end{equation}
Die zugrundeliegende Redoxreaktion ist in Gleichung (\ref{gl:kernreaktion}) dargestellt. Aus ihr geht der Zusammenhang zwischen dem Vorhandensein von Wasser und der Reduktion des Iodes hervor.
\begin{equation}\label{gl:kernreaktion}
	\ce{SO3^2- +I2 +H2O -> SO4^2- + 2H^+ +2I-}
\end{equation}
\subsection{Unterschied zwischen biamperometrischer und bivoltametrischer Titration}

