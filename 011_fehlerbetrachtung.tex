\section{Fehlerbetrachtung}
\label{sec:fehler}

Die zur Auswertung zugeteilten Werte waren zum Teil kaum auswertbar. Daher sind in den Tabellen des Kapitels \ref{sec:ergebnisse} sind einige Lücken zu erkennen. Die Auswertung der Messungen einiger Ethanol-Proben wurde unmöglich. Teilweise sind Messergebnisse auch nicht Nachvollziehbar. So zum Beispiel die enorme Differenz zwischen den Messungen der Isopropanol-Probe in \ref{tab:MesswerteIsopropanol}. Unklar ist warum man für den gleichen Stoff, 2-Propanol, auch Isopropanol genannt, zwei unterschiedliche Namen verwendet und sie in seperaten Versuchen untersucht hat. 

Dadurch, dass der Versuch nicht von den Autoren selbst durchgeführt wurde sind Rückschlüsse auf eventuell geschehene Missgeschicke oder grobe Anwendungsfehler, die die gefundenen Diskrepanzen erklären könnten, kaum mehr möglich. 

Messungenauigkeiten der Waage und der Mikroliterspritze können, wie auch Messfehler des Titrators, einen gewissen Fehler verursacht haben. Viel entscheidender erscheinen allerdings die Anwendungsfehler. Diese erstrecken sich von Übertragungsfehlern zu Abweichungen in der Handlungsreihenfolge oder der nicht-Einhaltung von vorgegebenen Zeiten und Hinweisen.

Auch verunreinigte oder überalterte Geräte und Chemikalien könnten Abweichungen verursacht haben.